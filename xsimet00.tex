%============================================================================
% tento soubor pouzijte jako zaklad
% (c) 2008 Michal Bidlo
% E-mail: bidlom AT fit vutbr cz
%============================================================================
% kodovaní: iso-8859-2 (zmena prikazem iconv, recode nebo cstocs)
%----------------------------------------------------------------------------
% zpracování: make, make pdf, make desky, make clean
% připomínky posílejte na e-mail: bidlom AT fit.vutbr.cz
% vim: set syntax=tex encoding=latin2:
%============================================================================
\documentclass[english,cover]{fitthesis} % odevzdani do wisu - odkazy, na ktere se da klikat
%\documentclass[cover,print]{fitthesis} % pro tisk - na odkazy se neda klikat
%\documentclass[english,print]{fitthesis} % pro tisk - na odkazy se neda klikat
%      \documentclass[english]{fitthesis}
% * Je-li prace psana v anglickem jazyce, je zapotrebi u tridy pouzit 
%   parametr english nasledovne:
% \documentclass[english]{fitthesis}
% * Neprejete-li si vysazet na prvni strane dokumentu desky, zruste 
%   parametr cover

% zde zvolime kodovani, ve kterem je napsan text prace
% "latin2" pro iso8859-2 nebo "cp1250" pro windows-1250, "utf8" pro "utf-8"
%\usepackage{ucs}
\usepackage[utf8]{inputenc}
\usepackage[T1, IL2]{fontenc}
\usepackage{url}
\DeclareUrlCommand\url{\def\UrlLeft{<}\def\UrlRight{>} \urlstyle{tt}}

%zde muzeme vlozit vlastni balicky
\usepackage{feynmp}


% =======================================================================
% balíček "hyperref" vytváří klikací odkazy v pdf, pokud tedy použijeme pdflatex
% problém je, že balíček hyperref musí být uveden jako poslední, takže nemůže
% být v šabloně
\ifWis
\ifx\pdfoutput\undefined % nejedeme pod pdflatexem
\else
  \usepackage{color}
  \usepackage[unicode,colorlinks,hyperindex,plainpages=false,pdftex]{hyperref}
  \definecolor{links}{rgb}{0.4,0.5,0}
  \definecolor{anchors}{rgb}{1,0,0}
  \def\AnchorColor{anchors}
  \def\LinkColor{links}
  \def\pdfBorderAttrs{/Border [0 0 0] }  % bez okrajů kolem odkazů
  \pdfcompresslevel=9
\fi
\fi

%Informace o praci/projektu
%---------------------------------------------------------------------------
\projectinfo{
  %Prace
  project=DP,            %typ prace BP/SP/DP/DR
  year=2015,             %rok
  date=\today,           %datum odevzdani
  %Nazev prace
  title.cs={3D Rekonstrukce historických míst z obrázků na Flickru},  %nazev prace v cestine
  title.en={3D Reconstruction of Historic Landmarks from Flickr Pictures}, %nazev prace v anglictine
  %Autor
  author={Vojtěch Šimetka},   %jmeno prijmeni autora
  author.title.p=Bc., %titul pred jmenem (nepovinne)
  %author.title.a=PhD, %titul za jmenem (nepovinne)
  %Ustav
  department=UPGM, % doplnte prislusnou zkratku: UPSY/UIFS/UITS/UPGM
  %Skolitel
  supervisor= Lukáš Polok, %jmeno prijmeni skolitele
  supervisor.title.p=Ing.,   %titul pred jmenem (nepovinne)
  %supervisor.title.a={Ph.D.},    %titul za jmenem (nepovinne)
  %Skolitel
  consultant= Viorela Simona Ila, %jmeno prijmeni skolitele
  consultant.title.p=Ing.,   %titul pred jmenem (nepovinne)
  consultant.title.a={Ph.D.},    %titul za jmenem (nepovinne)
  %Klicova slova, abstrakty, prohlaseni a podekovani je mozne definovat 
  %bud pomoci nasledujicich parametru nebo pomoci vyhrazenych maker (viz dale)
  %===========================================================================
  %Klicova slova
  keywords.cs={detekce klíčových bodů, extrakce bodů zájmu, párování bodů zájmu, počítačové vidění, monokulární vidění, stereo vidění, estimace pozice kamer, kalibrace kamer, structure from motion, bundle adjustment}, %klicova slova v ceskem jazyce
  keywords.en={keypoints detection, feature extraction, feature matching, computer vision, monocular vision, stereo vision, pose estimation, camera calibration, structure from motion, bundle adjustment}, %klicova slova v anglickem jazyce
  %Abstract
  abstract.cs={Tato práce popisuje problematiku návrhu a~vývoje aplikace pro~rekonstrukci 3D modelů z~2D obrazových dat, označované jako bundle adjustment. Práce analyzuje proces 3D rekonstrukce a~důkladně popisuje jednotlivé kroky. Prvním z~kroků je automatizované získání obrazové sady z~internetu. Je představena sada skriptů pro~hromadné stahování obrázků ze~služeb Flickr a~Google Images a~shrnuty požadavky na~tyto obrázky pro~co nejlepší 3D rekonstrukci. Práce dále popisuje různé detektory, extraktory a~párovací algoritmy klíčových bodů v~obraze s~cílem najít nejvhodnější kombinaci pro~rekonstrukci budov. Poté je vysvětlen proces rekonstrukce 3D struktury, její optimalizace a~jak je tato problematika realizovaná v~našem programu. Závěr práce testuje výsledky získané z~implementovaného programu pro~několik různých datových sad a~porovnává je s~výsledky ostatních podobných programů, představených v~úvodu práce.}, % abstrakt v anglickem jazyce
  abstract.en={This thesis describes challenges in design and development of an application which reconstructs 3D model given set of 2D images. This technique is called bundle adjustment. The thesi discusses the 3D reconstruction pipeline and elaborates on each step. The first step covers dataset acquisition from the internet. The scripts used to download such data from Flickr and Google Images are described and image characteristics necessary for a good reconstruction are identified. Hereafter the paper compares different feature detectors, extractors and matchers to find best suited combination for reconstruction of historic landmarks. This is followed by description the reconstruction and optimization steps and their implementation. At the end of the thesis the implemented solution is examined on several datasets and compared with other existing solutions presented at the very beginning of the thesis.}, % abstrakt v anglickem jazyce
  %Prohlaseni
  declaration={},
  %Podekovani (nepovinne)
  acknowledgment={} % nepovinne
}

%Abstrakt (cesky, anglicky)
%\abstract[cs]{Do tohoto odstavce bude zapsán výtah (abstrakt) práce v českém jazyce.}
%\abstract[en]{Do tohoto odstavce bude zapsán výtah (abstrakt) práce v anglickém jazyce.}

%Klicova slova (cesky, anglicky)
%\keywords[cs]{Sem budou zapsána jednotlivá klíčová slova v českém jazyce, oddělená čárkami.}
%\keywords[en]{Sem budou zapsána jednotlivá klíčová slova v anglickém jazyce, oddělená čárkami.}

%Prohlaseni
\declaration{Prohlašuji, že jsem tuto diplomovou práci vypracoval samostatně pod~vedením pana Ing.~Lukáše Poloka a~odborné konzultantky Ing.~Viorela Simona Ila, Ph. D. a~uvedl jsem všechny literární prameny a~publikace, ze~kterých jsem čerpal.}
%Další informace mi poskytli...
%Uvedl jsem všechny literární prameny a publikace, ze kterých jsem čerpal.}

%Podekovani (nepovinne)
\acknowledgment{Chtěl bych poděkovat mému vedoucímu diplomové práce Ing.~Lukáši Polokovi a~odborné konzultantce Ing.~Viorela Simona Ila, Ph.D. za~jejich trpělivost, ochotu, cenné rady, iniciativu a~hlavně za~čas, který do mě a~této práce investovali.}

\begin{document}
  % Vysazeni titulnich stran
  % ----------------------------------------------
  \maketitle
  % Obsah
  % ----------------------------------------------
  \setcounter{tocdepth}{1}
  \tableofcontents
  
  % Seznam obrazku a tabulek (pokud prace obsahuje velke mnozstvi obrazku, tak se to hodi)
  % \listoffigures
  % \listoftables 

  % Text prace
  % ----------------------------------------------
  %=========================================================================

\chapter{Introduction}
This chapter describes the motivation leading to the presentation of this paper and how is it related to the SLAM++ project at FIT VUT. The objectives of the project and the subjects included in this document are briefly explained. The chapter ends describing the overall structure and contents of the remaining of the paper.

\section{Motivation}
There is an immerse need to record and preserve our knowledge and perception of the world around us. Evidence of such tendency can be track thousands of years back in human existence as a cave paintings. Later we used more advanced techniques in form of written languages, sculptures, drawings, et cetera. Nowadays we have means to record our surroundings as we perceive it in 2D using cameras. However, it has proven difficult to process such information digitally. Even automated analysis of 2D information like typeset books is not a trivial problem and far from being mastered. While there are so called OCR converters available, in many cases they don't work reliably enough.

When it comes to 3D the problem gets much more difficult. Scanning 3D object reliably is nowadays possible in laboratory conditions, but there are hard limits like the size of the object, its structure or surface and material properties. Also the laboratory equipment used is much more expensive and physically larger compared to its 2D counterpart. This paper tries to address both of these problems by allowing user to create 3D model using multiple pictures of an object of interest from various sources. Such 3D model, even though it may be inaccurate, has number of applications. It can be used by archaeologists to preserve cultural heritage, by architects for spatial planning, by entertainers to create 3D models and virtual reality or by engineers to replicate existing 3D objects.

\section{Tools}
The process of creating a 3D model usually consist of two parts: scanning the object and reconstruction of the model. There are three main approaches how to scan physical object: contact, active non-contact  and passive non-contact scanners. The contact 3D scanners probe the subject through physical touch. The active non-contact scanners use a light in forms of laser or X-ray to scan the object while the passive non-contact scanners are using either multiple cameras, images with varying lighting conditions or silhouettes extruded from image with contrasted background. In this thesis we will be particularly interested in scanning objects using multiple images from various cameras. We will talk more about this in next chapter \ref{chapter:methodology} but for now note that for this we will use the SLAM++ framework \cite{www:slam}. The output of such scanning is usually a set of 3D points with a color information called point cloud. These data have to be filtered in order to remove noise, segmented and then reconstructed into polygonal model. The reconstruction can be done either manually using software like MeshLab \cite{www:meshlab} or automatized with framework like the PCL \cite{www:pcl}. 

\section{Objectives}
This paper aims to identify challenges of 3D reconstruction from a set of 2D images taken by various cameras leading to creation of a software that can do so automatically. The objective deals with the aim of reducing the correspondence problem between each two images and the study of camera modelling and calibration. An accurate estimation of the camera model and correspondence allows us to compute three-dimensional information from a two-dimensional image sequence. In order to eliminate artefacts the input set of images, obtained from internet, will have to be filtered to contain only daytime images from summer season.

The study of the geometry involved in multiple camera vision systems should allow us to present an application that can from a set of two-dimensional images reconstruct 3D scene depicted by the images.

\section{Overall Structure}
This paper consist of 4 chapters and a bibliography at the end of the document.

Chapter \ref{chapter:the-state-of-the-art} introduces reader to the process of estimation three-dimensional structure from two-dimensional image sequences. Firstly, it discuss existing solutions, like VisualSFM, photosynth and Bundler, elaborated on the output of these programs. Later they will be used as a benchmark for the implemented application. These will be used for the final application performance and effectiveness evaluation.. Then it focuses on the state-of-the-art features detectors, extractors and matchers, built in the SLAM++ frontend, and aims to compare theirs efficiency and performance. The comparison is then discussed and best combination in terms of performance and effectiveness chosen for the planned application.

In the chapter \ref{chapter:methodology} the whole 3D reconstruction process is thoroughly discussed and step by step explained. The steps that are already implemented, like dataset acquisition or feature detection, extraction and matching, are in detail described. The rest of the pipeline is also outlined to give the reader idea what are the next steps to be implemented. 

Finally, chapter \ref{chapter:conclusion} summarizes this document by discussing achieved goals.

\chapter{The State of the Art}
\label{chapter:the-state-of-the-art}
The following chapter presents to the reader the process of estimating three-dimensional structures from two-dimensional image sequences. After brief introduction into the pipeline, some of the existing solutions for the Structure from Motion (SfM) imaging technique are discussed. The programs described will be used as a benchmark for the final solution once it's presented. We take a closer look on first part of the pipeline consisting of the feature detectors, extractors and matchers. The goal is to select the best combination in terms of performance and effectiveness for reconstruction of historic landmarks. 

\section{Existing 3D Reconstruction Solutions}
The problem of creating 3D reconstruction from set of images has been addressed by many research groups. In this section we will talk about few of the widely known solutions. All of the programs below implement a subset of algorithm for structure from motion which consist of 7 distinct steps:
\begin{itemize}
	\item[1)] \textbf{Dataset aquisition.} First step in the SfM pipeline is selection of a input data. Specific requirements on the data varies throughout different software, however, we can generalize some properties of such set of images. The set has to contain images that are overlapping one another. Only such images are used in the reconstruction as they provide points seen by multiple cameras and therefore the 3D position can be calculated.
	\item[2)] \textbf{Keypoints detection.} Keypoints are parts of the image that are significant in some way. The significance is usually caused by a sudden change in gradient on relatively small part of the image. These points will be used to estimate the 3D representation. We will focus more on keypoints detection in section \ref{sec:detectors}. 
	\item[3)] \textbf{Feature extraction.} The detected keypoints are rarely used as provided by the detector as they do not provide enough information about the point itself. A set of calculations is applied in order to extract data from the surroundings of such point and enrich information about the keypoint. Resulting structure is called feature and contains all data required for next step. At this point the input image does not have to be kept in memory anymore.
	\item[4)] \textbf{Feature matching.} Now that we have keypoints represented as feature we want to establish a visual correspondence between a set of keypoints from two closely related images. This is done by so called feature matching and will be discussed in detail in section \ref{sec:matchers}.
	\item[5)] \textbf{Camera initialization and pose estimation.} Once we know correspondence between two closely related images, we can estimate the parameters of the camera(s). If not known, the camera matrix can be calculated and the relation between the points in 3D.
	\item[6)] \textbf{Structure computation.} Next step is to calculate the model's 3D structure from the points seen by different cameras. The quality of the structure is greatly affected by errors from previous steps.
	\item[7)] \textbf{Visualization.} Lastly the resulting 3D structure in form of point cloud is visualized.The visualization can be either just a set of images cleverly arranged to give impression of 3D, cloud of 3D points or polygonal model. In this thesis we will not be interested in the visualization of the resulting model, however we see this as an important part of the SfM pipeline.
\end{itemize}

\subsection*{Photosynth}
Photosynth is a software application developed by Microsoft. It is based on Photo Tourism, a research project by University of Washington graduate student Noah Snavely. Formerly the Photosynth was a 3D reconstruction software, however, in the current version output of the application is not a point cloud nor 3D model but an animation of morphing images or panorama. While it still works with images from various sources, the best result is achieved by importing photos from a single camera. Once imported, user has to choose the camera trajectory from four predefined options: spin, panorama, wall or walk. 

The Photosynth technology is using an interest point detection and matching algorithm developed by Microsoft Research which is similar in function to SIFT. Detected features are then matched between images and by analyzing subtle differences in the relationships between the features (angle, distance, etc.), the program identifies the 3D position of each feature, as well as the position and angle at which each photograph was taken. Everything is processed by Microsoft's servers and, once finished, pushed to the website or desktop/mobile application. There are little to none information about the whole process as this is a commercialized technology. \cite{www:photosynth}

\begin{figure}[ht]
	\begin{center}
		\includegraphics[keepaspectratio,width=\textwidth]{fig/Photosynth.png}
	\end{center}
	\caption{The Photosynth output of the Červená Lhota Castle in transition between several morphed images.}
	\label{fig:visualsfm}
\end{figure}

\subsection*{VisualSFM}
The Chungchang Wu's Visual Structure from Motion System is a GUI application for 3D reconstruction using structure from motion (SFM). The reconstruction system is modular and integrates several of other projects: SIFT on GPU(SiftGPU), Multicore Bundle Adjustment, and Towards Linear-time Incremental Structure from Motion. VisualSFM runs fast by exploiting multicore parallelism for feature detection, feature matching, and bundle adjustment. For dense reconstruction, the program supports Yasutaka Furukawa's PMVS/CMVS tool chain, and can prepare data for Michal Jancosek's CMP-MVS. In addition, the output of VisualSFM is natively supported by Mathias Rothermel and Konrad Wenzel's SURE.

The software follows the overall 3D reconstruction pipeline; It detects features using SIFT detector and SIFT extractor, matches feature pairs with the N-View Match, estimates the camera model for each image, removes images' distortion and then runs the dense reconstruction. Outputs of feature extraction and matching are stored as a binary files and are loaded if provided to save processing time. This enables use of other than built-in extractors and matcher, at least for the sparse reconstruction. \cite{www:visual_sfm}

\begin{figure}[ht]
	\begin{center}
		\includegraphics[keepaspectratio,width=\textwidth]{fig/VisualSFM.png}
	\end{center}
	\caption{The VisualSFM application GUI with sparse reconstruction of the Červená Lhota Castle.}
	\label{fig:visualsfm}
\end{figure}

\subsection*{Bundler}
Bundler is first well known Structure from Motion (SfM) system for unordered image collections from Noah Snavely. One of the first version of this Bundler system was used in the Photo Tourism project that was aqired by Microsoft and is now part of Photosynth. 

As other applications discussed, Bundler takes a set of images, image features, and image matches as input, and produces a 3D reconstruction of camera and sparse scene geometry as output. In order to get sparse point clouds, one has to run Bundler to get camera parameters, use the Bundle2PMVS program to convert the results into PMVS2 input and then run PMVS2. The system reconstructs the scene incrementally, a few images at a time, using a modified version of the Sparse Bundle Adjustment package of Lourakis and Argyros as the underlying optimization engine. Bundler has been successfully run on many Internet photo collections, as well as more structured collections. \cite{www:bundler}

\section{Detectors}
\label{sec:detectors}
A successful 3D reconstruction stands and falls on good features detection.The quality and robustness of features is (usually) much more important then their quantity which will be demonstrated by the Dense detector later in this section. The ideal feature detector finds salient image regions such that they are repeatedly detected despite change of viewpoint; more generally it is robust to all possible image transformations. Therefore, it does not detect any points in uniform and uninteresting surfaces like sky or texture-less walls. The best detector to be used depends heavily on the requested task. In our application features we are interested in are edges and corners of buildings and their distinct parts.

We can divide types of image features into following categories (please note that a detector can detect features from multiple categories):
% CITATION NEEDED!!!
\begin{itemize}
	\item \textbf{Edge} is a point where there is a sudden change between adjacent pixels (strong gradient magnitude). Generally an edge can be of almost any arbitrary shape and may include junctions. Locally edges have a one-dimensional structure.
	\item \textbf{Interest point} has a local two dimensional structure. We can think of it as two-dimensional edge, in fact early algorithms were used to detect interest points as edges and then selected the interest points by further calculation. In some literature you the interest points may be referred to as corners.
	\item \textbf{Blobs} provide a complementary description of image structures in terms of regions, as opposed to corners that are more point-like. A term regions of interest or interest points are sometimes used as the blob descriptors often contain a preferred point (a local maximum or a center of gravity). Blobs allows detection of smooth areas in an image that might not be detected as an edge or corner.
	\item \textbf{Ridges} are in computer vision a set of curves whose points are have a local maximum in at least one dimension. This notion captures the intuition of geographical ridges. Ridge detection is usually much harder then Edge, Interest point or Blob detection.
\end{itemize}

In the remainder of this section feature detectors implemented in the SLAM++ frontend using OpenCV will be presented. Each detector will be run on the same set of images with historic landmarks in order to evaluate the effectiveness and speed. Then a summary of the results will be presented.
 
\begin{itemize}
	\item The \textbf{Robust Invariant Scalable Keypoints (BRISK)} detector uses scale-space pyramid layers of octaves and intra-octaves to detect corners in an image. The algorithm uses FAST feature detector score and was developed to get the better of SIFT and SURF detectors. However, in our case the performance gain is not worth decreased feature quality. \cite{article:brisk}
	
	\item \textbf{Dense Sampling} uses a regular grid to find a keypoints in the image. This results in good coverage of the entire object or scene and a constant amount of feature per image area. The dense sampling is fast as the detector selects all points on a grid without analysis of the surrounding. On the downside, dense sampling cannot reach the same level of repeatability as obtained with interest points, unless sampling is performed extremely densely, but then the number of features quickly grows unacceptably large. The dense sampling is therefore not useful in the SfM model estimation, but can be used for a dense reconstruction once sparse structure is calculated. \cite{article:dense}
	
	\item The \textbf{Features from Accelerated Segment Test (FAST)} aims to rapidly increase performance of feature detection while sustaining feature quality of SIFT-like detectors. The algorithm detects corners in the image and should be used with SIFT or SURF extractor for best performance. As the FAST select in our case three times more features hundred times faster than SIFT (resp. 50 times faster than SURF) we market this as one of the interesting detectors for the final implementation. \cite{article:fast}
	
	\item One of the most known feature detectors is the \textbf{Harris Corner Detector} . It can identify similar regions between images that are related through affine transformations and have different illuminations. Even though the Harris Corner Detector is fast, it does not select enough keypoints and therefore is not suitable for the 3D building reconstruction. \cite{www:harris}
	
	\item The \textbf{Good Features to Track (GFTT)} detector is modified version of the Harris Corner Detector described earlier. It is still classified as a corner detector, however, the scoring function differs. Compared to the Harris, the algorithm was slightly slower, with higher amount of features. Nevertheless, both of these algorithms do not perform well enough for our problem. \cite{article:gftt}
	
	\item The \textbf{Oriented FAST and Rotated BRIEF (ORB)} detector originated from the OpenCV Labs. It's goal was to offer robustness of a SIFT and SURF, while maintaining fast processing time like FAST and BRIEF combination. While this may be true, for our problem the ORB detector does not perform well enough. The feature found rarely belong to a building and usually chunks around trees and vegetation. \cite{www:orb}\cite{article:orb}
	
	\item The \textbf{Maximally Stable Extremal Regions} is a blob detector. For our task this detector performs poorly and takes even more time than SIFT detector.
	
	\item A \textbf{Scale Invariant Feature Transform (SIFT)} keypoint is image region with an orientation. The detector uses as a keypoints image structures  which resemble blobs. The use of the detector is licenced which is one of the reasons why we would like to use a different detecter. However, as expected, the detector performs very well and is used by other SfM software we discussed earlier. \cite{article:sift}
	
	\item The \textbf{Speeded Up Robust Features (SURF)} detector is modification of the SIFT detector. It addresses slow processing of the SIFT while maintaining reasonable efficiency. While it can surely be used in the SfM application, from our experiments we discovered that the increased performance greatly decreases feature detection for (in our case) important structures.  \cite{www:surf}
\end{itemize}

We've implemented all of the standard OpenCV keypoints detectors from previous list in the SLAM++ frontend. The goal was to test how do they perform on our specific problem: detecting keypoints in buildings. The figure \ref{fig:detectors} shows three detectors (SIFT, SURF and FAST) that are suitable for our task as they find enough relevant features in an image. 

\begin{figure}[ht]
	\begin{center}
		\includegraphics[keepaspectratio,width=\textwidth]{fig/detectors.pdf}
	\end{center}
	\caption{Results of the feature detection evaluation on as set of 500 various images from the Červená Lhota dataset. Graph a) shows average time necessary for processing an image using selected detector. In graph b) you can find how many features on average were detected in a single image.}
	\label{fig:detectors}
\end{figure}

\section{Extractors}
\label{sec:extractors}
In order to work further with the keypoints detected in previous step, the keypoints have to be analysed and transformed into so called feature descriptor. The process consist of inspecting local image patch around the keypoint to be extracted. This extraction may involve quite considerable amounts of image processing and involves reducing the amount of resources required to describe a the original data. The result is known as a feature descriptor or a feature vector. Among the information that may be stored within feature descriptor, one can mention local histograms. In addition to such attribute information, the keypoints detection step may also provide complementary attributes, such as the edge orientation, gradient magnitude in edge detection and the polarity or the strength of the blob in blob detection. The authors of detectors usually specify which extractor should work best for their detection algorithm, some even provide their own. Nevertheless, we tried all combinations with detectors to get the best results for our application.

There are two types of descriptors in the OpenCV that are now available in Slam++ frontend; a) descriptors using floating point and b) descriptors storing information as a binary data in unsigned char type.
\begin{itemize}
	\item[a)] \textbf{Float} descriptors:
	
	\begin{itemize}
		\item \textbf{SIFT:} The scale-invariant feature transform of a neighbourhood is a 128-dimensional vector of histograms of image gradients. The region, at the appropriate scale and orientation, is divided into a $4\times 4$ square grid, each cell of which yields a histogram with 8 orientation bins. The SIFT extractor is advised to be used with the SIFT, SURF and FAST detector.
		\item \textbf{SURF:} The speeded up robust feature extractor uses either 128 or 64-dimensional vector of histograms of image gradients.An oriented quadratic grid of $4 \times 4$ square sub-regions is laid over the keypoint and a wavelet response computed for each square. According to literature the SIFT, SURF and FAST detector can be used with the SURF extractor.
	\end{itemize}
	
	\item[b)] \textbf{Binary} descriptors:
	
	\begin{itemize}
		\item \textbf{BRIEF:} The Binary Robust Independent Elementary Feature descriptor is a 128, 256 or 512-dimensional bitstring which is a good compromise between speed, storage efficiency and recognition rate. The descriptor is much smaller (16, 32 or 64 bytes) compared to floating point descriptors, while maintaining a good performance compared to SURF or U-SURF. \cite{article:brief}
		
		\item \textbf{ORB:} Unlike BRIEF, Oriented FAST and Rotated BRIEF (ORB) is comparatively scale and rotation invariant while still employing the very efficient Hamming distance metric for matching. As such, it is preferred for real-time applications, but may be suitable for some offline applications as well. \cite{www:orb}\cite{article:orb}
		
		\item \textbf{FREAK:} The Fast Retina Keypoint extractor aims to be faster and more robust than SIFT and SURF. It uses a novel keypoint descriptor inspired by the human visual system to compute cascade of binary strings.\cite{article:freak}
		
		\item \textbf{BRISK:}  The  Binary Robust Invariant Scalable Keypoints extractor uses a 64-byte binary descriptor composed as a binary string by concatenating the results of simple brightness comparison tests. \cite{article:brisk}
	\end{itemize}
\end{itemize}

\section{Matchers}
\label{sec:matchers}

\chapter{Methodology}
\label{chapter:methodology}
This chapter describes the process of 3D reconstruction from a set of images to the point cloud output. It puts the features detection, extraction and matching into context of 3D reconstruction. In order to select only inliers, the RUNSAC algorithmes filters the matched feature pairs. Then, using mathematical apparatus from chapter  the fundamental matrix and camera model are estimated. Finally, once we know the camera model, we can run dense reconstruction, creating the resulting point cloud.  
\section{Three-dimensional Structure Estimation Pipeline}
\subsection*{Selecting Dataset}
\subsection*{Feature Detection and Extraction}
\subsection*{Feature Matching}
\section{3D Reconstruction Approaches}
Stereo, Mono, Uncalibrated
frontend and backend

\chapter{Conclusion}
\label{chapter:conclusion}

%=========================================================================
 % viz. obsah.tex

  % Pouzita literatura
  % ----------------------------------------------
\ifczech
  \bibliographystyle{czechiso}
\else 
  \bibliographystyle{plain}
%  \bibliographystyle{alpha}
\fi
  \begin{flushleft}
  \bibliography{literatura} % viz. literatura.bib
  \end{flushleft}
  \appendix
  
  \chapter{Content of the DVD}
\begin{itemize}
	\item \textbf{thesis.pdf} - PDF version of the thesis text.
	\item \textbf{tex/} - \LaTeX ~source codes of the thesis text.
	\item \textbf{src/} - Source codes of our program described in chapter~\ref{chapter:implementation}.
	\item \textbf{datasets/} - Collected image datasets described in the section~\ref{sec:experiments-datasets} and some other.
	\item \textbf{scripts/} - Some of the scripts used in evaluation of our programme.
	\item \textbf{experiments/} - Experimental results used in chapter~\ref{chapter:experiments}.
	\item \textbf{third\textunderscore party/} - Other Structure from Motion and Bundle Adjustment application sources.
	\item \textbf{install.sh} - Installation script (requires OpenCV and cmake to be installed).
	\item \textbf{README} - Read me file describing the program interface and simple samples.
	\item \textbf{bin/} - Location of a binary version of the program after running \texttt{install.sh}.
\end{itemize}
\chapter{Poster}
\begin{figure}[h]
	\begin{center}
		\includegraphics[keepaspectratio,width=10cm]{fig/poster.pdf}
	\end{center}
	\caption{Preview of poster presenting our program.}
	\label{fig:posteri}
\end{figure}

\chapter{Dense Reconstruction with VisualSFM}
\label{visualsfm:reconstruction}
The VisualSFM is an amazing collection of software that can create dense reconstruction from random images of a static scene. The application has a graphical user interface depicted in figure~\ref{fig:visualsfm-gui}.
\begin{figure}[h]
	\begin{center}
		\includegraphics[keepaspectratio,width=\textwidth]{fig/visualsfm-gui.pdf}
	\end{center}
	\caption{GUI of the VisualSFM with few important buttons.}
	\label{fig:visualsfm-gui}
\end{figure}
\section{Reconstruction Process}
The whole process of reconstruction follows:
\begin{itemize}
	\item[1.] Choose \textbf{File} $\rightarrow$ \textbf{Open+ Multi Images}, navigate yourself to the dataset and select desired photos. After a while the photos should load into the system and should be displayed in a matrix.
	\item[2.] Next we need to compute keypoints and feature correspondences. Click \includegraphics[keepaspectratio,width=.5cm]{fig/visualsfm-compute-matches.png} or choose \textbf{SfM} $\rightarrow$ \textbf{Pairwise Matching} $\rightarrow$ \textbf{Compute Missing Match}. The system will now detect keypoints and calculate matches for all new photos. These feature and matches are implicitly cached in separate files \texttt{.sift} and \texttt{.mat}. Please note that this step may take a significant amount of time. The progress is shows in the \textbf{Log Window}.
	\item[3.] Click on the button \includegraphics[keepaspectratio,width=.5cm]{fig/visualsfm-sparse-reconstruction.png} or choose \textbf{SfM} $\rightarrow$ \textbf{Reconstruct Sparse} to compute sparse reconstruction. The view should change slightly showing the reconstructed 3D scene. The program may not create single model for the input dataset, but you can browse distinct models by pressing up and down arrow keys (model number is indicated in the window name between first pair of square brackets).
	\item[4.] Last step is running the dense reconstruction. Click \includegraphics[keepaspectratio,width=.5cm]{fig/visualsfm-dense-reconstruction.png} or choose \textbf{SfM} $\rightarrow$ \textbf{Reconstruct Dense}. You will be prompted to choose working directory and once all the files are saved to this directory, the PMVS2 dense reconstruction starts. Note that the dense reconstruction is a complex problem and takes a lot of time. Once finished, you can toggle between sparse and dense with hotkey \textbf{tabulator}. An example of sparse and dense reconstruction is in the figure~\ref{fig:visualsfm-sparse-dense}.
\end{itemize}
The output, dense structure of each model is stored in folder \texttt{models} in a PLY format. The estimated camera calibrations can be found in the text file \texttt{cameras\textunderscore v2.txt}. The camera centres are stored in the PLY format as  \texttt{centers-[model\textunderscore ID].ply} containing the 3D locations for every camera in each model.
\begin{figure}[h]
	\begin{center}
		\includegraphics[keepaspectratio,width=\textwidth]{fig/visualsfm-sparse-dense.pdf}
	\end{center}
	\caption{Sparse (left) and dense (right) reconstruction of the Model House dataset.}
	\label{fig:visualsfm-sparse-dense}
\end{figure}

\section{Useful Tips and Controls}
The VisualSFM offers many hidden functionality useful for reconstructing the 3D scene. Full list can be found in the VisualSfM online documentation\footnote{VisualSfM documentation \url{http://ccwu.me/vsfm/doc.html}}.
\begin{itemize}
	\item \textbf{Camera calibration.} If camera calibration is known, it can be provided to the program by choosing \textbf{SfM} $\rightarrow$ \textbf{More Functions} $\rightarrow$ \textbf{Set Fixed Calibration}. This calibration can be also made shared across all cameras.
	
	\item \textbf{Selecting Initial Pair.} Initial camera pair for the reconstruction can be chosen by selecting the pair with left and right arrow keys and then choosing \textbf{SfM} $\rightarrow$ \textbf{More Functions} $\rightarrow$ \textbf{Set Initialization Pair}.
	
	\item \textbf{Keyboard shortcuts and mouse controls.}
	
	\begin{tabular}{l l}
		\textbf{Mouse middle wheel} & Zoom the scene. \\
		\textbf{Right mouse drag} & Rotate the scene. \\
		\textbf{Left mouse drag} & Pan the scene. \\
		\textbf{Tabulator} & Switch between sparse and dense reconstruction. \\
		\textbf{Up/Down} & Switch between different models. \\
		\textbf{Left/Right} & Switch between camera pairs. \\
		\textbf{T} & Switch between visualization modes: \begin{tabular}{l}
			\text{1) cameras + 3D points} \\
			\text{2) cameras only} \\
			\text{3) 3D points only} \\
		\end{tabular}\\
	\end{tabular}
\end{itemize}

\chapter{Dense to Textured Surface Reconstruction using Meshlab and Blender}
\label{app:surface-reconstruction}
This chapter describes the process of creation textured 3D polygonal model from dense point cloud. It assume that the user have a 3D dense reconstruction in a format produced by VisualSFM. The process start in MeshLab:
\begin{itemize}
	\item[1.] In MeshLab, select \textbf{File} $\rightarrow$ \textbf{OpenProject} and choose file \texttt{bundle.rd.out} in the dense reconstruction data folder. Next prompt requires the \texttt{list.txt} from the same folder. 
	\item[2.] Right now the sparse point cloud is opened which is not what we want. Click the icon \includegraphics[keepaspectratio,width=.5cm]{fig/meshlab-layers.png} or \textbf{View} $\rightarrow$ \textbf{Show Layer Dialog}. A new toolbar will appear on the right of the screen. Click on the model with the right mouse button and from the context menu select \textbf{Delete Current Mesh} (as depicted in figure~\ref{fig:meshlab-1}).
\begin{figure}[h]
	\begin{center}
		\includegraphics[keepaspectratio,width=9cm]{fig/meshlab-1.pdf}
	\end{center}
	\caption{How to erase sparse cloud mesh in MeshLab.}
	\label{fig:meshlab-1}
\end{figure}
	\item[3.] Choose \textbf{File} $\rightarrow$ \textbf{Import Mesh} and select adequate model from folder \texttt{models}.
	\item[4.] The model shown is now a dense point cloud, but it is almost certainly quite noisy. We can remove some of the noisy data by selecting them with tool \includegraphics[keepaspectratio,width=.5cm]{fig/meshlab-select.png} (\textbf{Edit} $\rightarrow$ \textbf{Select Vertices}), and erasing them with \includegraphics[keepaspectratio,width=.5cm]{fig/meshlab-delete.png}.

	\item[5.] Next step is to create a surface reconstruction. Choose \textbf{Filters} $\rightarrow$ \textbf{Point Set} $\rightarrow$ \textbf{Surface Reconstruction: Poisson}. We suggest you to use following values:
	
	\begin{tabular}{l l}
		Octree Depth & 12 \\
		Solver Divide & 10 \\
		Samples per Node & 1 \\
		Surface offsetting & 1 \\
	\end{tabular}
	
%\begin{figure}[ht]
%	\begin{center}
%		\includegraphics[keepaspectratio,width=\textwidth]{fig/meshlab-2.pdf}
%	\end{center}
%	\caption{The process of manually erasing incorrect points.}
%	\label{fig:meshlab-2}
%\end{figure}

	\item[6.] However, the surface reconstruction creates a lots of incorrect faces. We once again erase the vertices using tools from step 4. Once done, choose \textbf{Filters} $\rightarrow$ \textbf{Selection} $\rightarrow$ \textbf{Select non-manifold Edges}, hit apply and erase these points with vertex erase tool \includegraphics[keepaspectratio,width=.5cm]{fig/meshlab-delete.png}.
	
	\item[7.] Now we finally apply textures. Select \textbf{Filters} $\rightarrow$ \textbf{Texture} $\rightarrow$ \textbf{Parametrization + texturing from registered rasters}. Make sure the following options are checked: \textbf{Color correction, Use distance weight, Use image border weight, Clean isolated triangles} and \textbf{UV stretching}.
	\item[8.] The last step in the MeshLab tool is to export the model to the file format supported by Blender. Select \textbf{File} $\rightarrow$ \textbf{Export mesh as} and make sure you save it as an \texttt{obj} type.
\end{itemize}

The next part is more or less optional and contains mostly just storing the texture in the file itself.
\begin{itemize}
	\item[1.] Lets open Blender, right click on the cube, hit key \textbf{x} and click delete.
	\item[2.] Choose \textbf{File} $\rightarrow$ \textbf{Import} $\rightarrow$ \textbf{Wavefront (.obj)}.
	\item[3.] On the panel on the right select texture tab. Add new texture and change the type of the texture to \textbf{Image or Movie}. Ten press \textbf{Open} and load the appropriate texture file. If necessary the UV mapping coordinates can be altered, but this is beyond the scope of our tutorial.
	\item[4.] Now we can export the 3D model to some reasonable 3D file format, for example the\texttt{fbx}.
\end{itemize} 


\begin{figure}[ht]
	\begin{center}
		\includegraphics[keepaspectratio,width=\textwidth]{fig/reconstruction-overview.pdf}
	\end{center}
	\caption{Distinct reconstruction phases from sparse reconstruction to textured 3D model.}
	\label{fig:reconstruction-overview}
\end{figure}

%\chapter{Manual}
%\chapter{Konfigrační soubor}
%\chapter{RelaxNG Schéma konfiguračního soboru}
%\chapter{Plakat}
 % viz. prilohy.tex
\end{document}
